\documentclass{beamer}

% Input all common stuff
\input{common_predoc}

\subtitle{Session 3: More data types}

\begin{document}

\frame{\titlepage}

\begin{frame}
\frametitle{Table of Contents}
\tableofcontents
\end{frame}


\section{Arrays}

\subsection{Declaring arrays}

\begin{frame}[fragile]
	\frametitle{Arrays}
	
	\defiblock{array}{a series of elements of the same type occupying a contiguous block of memory.}
	
	Format for declaring an array is:
	\begin{lstlisting}
	type name[num_elements];
	\end{lstlisting}
	Where \texttt{type} is any valid data type and \texttt{num\_elements} is a constant positive integer.
	\pause
	Some examples:
	\begin{lstlisting}
	unsigned int lotteryNumbers[7];
	
	double planetMasses[8];
	
	const unsigned int numParticles = 128;
	double xPositions[numParticles];
	double yPositions[numParticles];
	\end{lstlisting}
	Last example shows how we can use constant variable as array size.
	
\end{frame}

\subsection{Using arrays}

\begin{frame}[fragile]
  \frametitle{Initialising arrays}
  When declaring an array it can be initialised as follows:
  \begin{lstlisting}
unsigned int lotteryNumbers[7] = {16, 3, 28, 9, 24, 10, 8}
  \end{lstlisting}
  the size can be left out, in which case the number of values given is used:
  \begin{lstlisting}
unsigned int lotteryNumbers[] = {16, 3, 28, 9, 24, 10, 8}
  \end{lstlisting}  
\end{frame}

\begin{frame}[fragile]
  \frametitle{Accessing elements}
  
  To access an element of an array the format is:
  \begin{lstlisting}
  name[index]
  \end{lstlisting}
  \warnblock{In C++ array numbering starts at 0!  This is a huge source of confusion especially if you're used to a programming language like Fortran where arrays start at 1.}
  \pause
  For example, to read the 3$^{rd}$ lottery number use:
  \begin{lstlisting}
  unsigned int third = lotteryNumbers[2];
  \end{lstlisting}
  \pause
  To write the 3$^{rd}$ lottery number use:
  \begin{lstlisting}
  lotteryNumber[2] = 23;
  \end{lstlisting}
  \pause
  You can also access the elements using a variable:
  \begin{lstlisting}
  for(int i = 0; i < 7; ++i)
     std::cout << lotteryNumbers[i] << " ";
  \end{lstlisting}
\end{frame}

\subsection{Multidimensional arrays}

\begin{frame}[fragile]
  \frametitle{Multidimensional arrays}
  \framesubtitle{Because with only two friends 1D is sooo boring}
  Think of multidimensional arrays as beging "arrays of arrays".  An example:
  \lstinputlisting[language=C++,title=\lstsrctitle,linerange={10-14,21-32}]{../code/2_more_data_types/lectures/centre_of_mass.cpp}  
  

\end{frame}

\section{Char sequences and strings}


\frame {
  \frametitle{Char, string}
}

\section{Pointers and references}

\frame {
  \frametitle{Pointers and references}
}


\end{document}