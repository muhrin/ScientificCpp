\documentclass{beamer}

% Input all common stuff


% THEME SETTINGS %%%%%%%%%%%%
\usetheme[secheader]{Boadilla}
\usecolortheme{default}
\useinnertheme{circles}
\setbeamertemplate{blocks}[default]

% Get rid of bottom navigation bars
\setbeamertemplate{footline}[page number]{}

% Gets rid of navigation symbols
\setbeamertemplate{navigation symbols}{}

% PACKAGES %%%%%%%%%%%%%%%%%%%
\usepackage{listings}
\usepackage{xcolor}
\usepackage{multirow}
\usepackage{array}
\usepackage{verbatim}
\usepackage{pbox}

% LISTINGS STYLE %%%%%%%%%%%%%
\definecolor{bluekeywords}{rgb}{0.13,0.13,1}
\definecolor{greencomments}{rgb}{0,0.5,0}
\definecolor{redstrings}{rgb}{0.9,0,0}
\definecolor{listingsbg}{HTML}{EFEFFF}

\lstset{language=C++,
  showspaces=false,            % show spaces adding particular underscores
  showtabs=false,              % show tabs within strings adding particular underscores
  breaklines=true,             % sets automatic line breaking
  showstringspaces=false,      % underline spaces within strings
  breakatwhitespace=true,      % sets if automatic breaks should only happen at whitespace
  captionpos=b,                % sets the caption-position to bottom
  escapeinside={(*@}{@*)},
  commentstyle=\color{greencomments},
  keywordstyle=\color{bluekeywords}\bfseries,
  stringstyle=\color{redstrings},
  basicstyle=\ttfamily\fontsize{9}{10}\selectfont,
  backgroundcolor=\color{listingsbg},
}

% MACROS %%%%%%%%%%%%%%%%%%%%%
\newcommand{\lstsrctitle}[0]{%
 {\scriptsize\lstname}
}

\newcommand{\kw}[1]{\texttt{\textcolor{bluekeywords}{\ttfamily\fontsize{9}{10}\selectfont #1}}}

\newenvironment{varblock}[3]{%
  \setbeamercolor{block body}{#2}
  \setbeamercolor{block title}{#3}
  \begin{block}{#1}}{\end{block}}

% Colours taken from http://twitter.github.com/bootstrap/components.html#alerts
% Do block %%%%%%%%%%%%%%%%%%%%%%
\definecolor{doblocktext}{HTML}{468847}
\definecolor{doblockbg}{HTML}{DFF0D8}
\newenvironment{doblocke}[1]{% begin
  \begin{varblock}{\textbf{Do}}%
  {bg=doblockbg,fg=doblocktext}{bg=doblocktext,fg=white}%
  \lstset{backgroundcolor=}% The background should be the same as that of the block
  #1}%
  {\end{varblock}} %end
\newcommand{\doblock}[1]{%
 \begin{doblocke}#1\end{doblocke} 
}

% Don't block %%%%%%%%%%%%%%%%%%%
\definecolor{dontblocktext}{HTML}{B94A48}
\definecolor{dontblockbg}{HTML}{F2DEDE}  
\newenvironment{dontblocke}[1]{% begin
  \begin{varblock}{\textbf{Don't}}%
  {bg=dontblockbg,fg=dontblocktext}{bg=dontblocktext,fg=white}
  \lstset{backgroundcolor=}% The background should be the same as that of the block
  #1}%
  {\end{varblock}} %end

\newcommand{\dontblock}[1]{%
 \begin{varblock}{\textbf{Don't}}{bg=dontblockbg,fg=dontblocktext}{bg=dontblockbg,fg=dontblocktext}
 #1
 \end{varblock} 
}

\definecolor{defiblocktext}{HTML}{3A87AD}
\definecolor{defiblockbg}{HTML}{D9EDF7}
\newcommand{\defiblock}[2]{%
 \begin{varblock}{\textbf{Definition}}%
 {bg=defiblockbg,fg=defiblocktext}{bg=defiblocktext,fg=white}
  \begin{tabular}{ll}\textit{#1} & #2\end{tabular}
 \end{varblock} 
}

\definecolor{warnblocktext}{HTML}{C09853}
\definecolor{warnblockbg}{HTML}{D9EDF7}
\newcommand{\warnblock}[1]{%
 \begin{varblock}{\textbf{Warning!}}{bg=warnblockbg,fg=warnblocktext}{bg=warnblockbg,fg=warnblocktext}
  #1
 \end{varblock} 
}

\newcommand{\cout}[1]{%
 Output: \pbox[t]{\textwidth}{\ttfamily\fontsize{9}{10}\selectfont{}#1}
}

% CONSTANTS %%%%%%%%%%%%%%%%%%%%%%%%%%%


% Common title slide stuff
\title{Introduction to Programming with Scientific C++}
\author{Martin Uhrin}
\institute{UCL}
\date{November 7-9th 2012}

\subtitle{Session 3: More data types}

\begin{document}

\frame{\titlepage}

\begin{frame}
\frametitle{Table of Contents}
\tableofcontents
\end{frame}


\section{Arrays}

\subsection{Declaring arrays}

\begin{frame}[fragile]
	\frametitle{Arrays}
	
	\defiblock{array}{a series of elements of the same type occupying a contiguous block of memory.}
	
	Format for declaring an array is:
	\begin{lstlisting}
	type name[num_elements];
	\end{lstlisting}
	Where \texttt{type} is any valid data type and \texttt{num\_elements} is a constant positive integer.
	\pause
	Some examples:
	\begin{lstlisting}
	unsigned int lotteryNumbers[7];
	
	double planetMasses[8];
	
	const unsigned int numParticles = 128;
	double xPositions[numParticles];
	double yPositions[numParticles];
	\end{lstlisting}
	Last example shows how we can use constant variable as array size.
	
\end{frame}

\subsection{Using arrays}

\begin{frame}[fragile]
  \frametitle{Initialising arrays}
  When declaring an array it can be initialised as follows:
  \begin{lstlisting}
unsigned int lotteryNumbers[7] = {16, 3, 28, 9, 24, 10, 8}
  \end{lstlisting}
  the size can be left out, in which case the number of values given is used:
  \begin{lstlisting}
unsigned int lotteryNumbers[] = {16, 3, 28, 9, 24, 10, 8}
  \end{lstlisting}  
\end{frame}

\begin{frame}[fragile]
  \frametitle{Accessing elements}
  
  To access an element of an array the format is:
  \begin{lstlisting}
  name[index]
  \end{lstlisting}
  \warnblock{In C++ array numbering starts at 0!  This is a huge source of confusion especially if you're used to a programming language like Fortran where arrays start at 1.}
  \pause
  For example, to read the 3$^{rd}$ lottery number use:
  \begin{lstlisting}
  unsigned int third = lotteryNumbers[2];
  \end{lstlisting}
  \pause
  To write the 3$^{rd}$ lottery number use:
  \begin{lstlisting}
  lotteryNumber[2] = 23;
  \end{lstlisting}
  \pause
  You can also access the elements using a variable:
  \begin{lstlisting}
  for(int i = 0; i < 7; ++i)
     std::cout << lotteryNumbers[i] << " ";
  \end{lstlisting}
\end{frame}

\subsection{Multidimensional arrays}

\begin{frame}[fragile]
  \frametitle{Multidimensional arrays}
  \framesubtitle{Because with only two friends 1D is sooo boring}
  Think of multidimensional arrays as beging "arrays of arrays".  An example:
  \lstinputlisting[language=C++,title=\lstsrctitle,linerange={10-14,21-32}]{../code/2_more_data_types/lectures/centre_of_mass.cpp}  
  

\end{frame}

\section{Char sequences and strings}


\frame {
  \frametitle{Char, string}
}

\section{Pointers and references}

\frame {
  \frametitle{Pointers and references}
}


\end{document}