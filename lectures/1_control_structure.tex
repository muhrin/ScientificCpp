\documentclass{beamer}

% Input all common stuff


% THEME SETTINGS %%%%%%%%%%%%
\usetheme[secheader]{Boadilla}
\usecolortheme{default}
\useinnertheme{circles}
\setbeamertemplate{blocks}[default]

% Get rid of bottom navigation bars
\setbeamertemplate{footline}[page number]{}

% Gets rid of navigation symbols
\setbeamertemplate{navigation symbols}{}

% PACKAGES %%%%%%%%%%%%%%%%%%%
\usepackage{listings}
\usepackage{xcolor}
\usepackage{multirow}
\usepackage{array}
\usepackage{verbatim}
\usepackage{pbox}

% LISTINGS STYLE %%%%%%%%%%%%%
\definecolor{bluekeywords}{rgb}{0.13,0.13,1}
\definecolor{greencomments}{rgb}{0,0.5,0}
\definecolor{redstrings}{rgb}{0.9,0,0}
\definecolor{listingsbg}{HTML}{EFEFFF}

\lstset{language=C++,
  showspaces=false,            % show spaces adding particular underscores
  showtabs=false,              % show tabs within strings adding particular underscores
  breaklines=true,             % sets automatic line breaking
  showstringspaces=false,      % underline spaces within strings
  breakatwhitespace=true,      % sets if automatic breaks should only happen at whitespace
  captionpos=b,                % sets the caption-position to bottom
  escapeinside={(*@}{@*)},
  commentstyle=\color{greencomments},
  keywordstyle=\color{bluekeywords}\bfseries,
  stringstyle=\color{redstrings},
  basicstyle=\ttfamily\fontsize{9}{10}\selectfont,
  backgroundcolor=\color{listingsbg},
}

% MACROS %%%%%%%%%%%%%%%%%%%%%
\newcommand{\lstsrctitle}[0]{%
 {\scriptsize\lstname}
}

\newcommand{\kw}[1]{\texttt{\textcolor{bluekeywords}{\ttfamily\fontsize{9}{10}\selectfont #1}}}

\newenvironment{varblock}[3]{%
  \setbeamercolor{block body}{#2}
  \setbeamercolor{block title}{#3}
  \begin{block}{#1}}{\end{block}}

% Colours taken from http://twitter.github.com/bootstrap/components.html#alerts
% Do block %%%%%%%%%%%%%%%%%%%%%%
\definecolor{doblocktext}{HTML}{468847}
\definecolor{doblockbg}{HTML}{DFF0D8}
\newenvironment{doblocke}[1]{% begin
  \begin{varblock}{\textbf{Do}}%
  {bg=doblockbg,fg=doblocktext}{bg=doblocktext,fg=white}%
  \lstset{backgroundcolor=}% The background should be the same as that of the block
  #1}%
  {\end{varblock}} %end
\newcommand{\doblock}[1]{%
 \begin{doblocke}#1\end{doblocke} 
}

% Don't block %%%%%%%%%%%%%%%%%%%
\definecolor{dontblocktext}{HTML}{B94A48}
\definecolor{dontblockbg}{HTML}{F2DEDE}  
\newenvironment{dontblocke}[1]{% begin
  \begin{varblock}{\textbf{Don't}}%
  {bg=dontblockbg,fg=dontblocktext}{bg=dontblocktext,fg=white}
  \lstset{backgroundcolor=}% The background should be the same as that of the block
  #1}%
  {\end{varblock}} %end

\newcommand{\dontblock}[1]{%
 \begin{varblock}{\textbf{Don't}}{bg=dontblockbg,fg=dontblocktext}{bg=dontblockbg,fg=dontblocktext}
 #1
 \end{varblock} 
}

\definecolor{defiblocktext}{HTML}{3A87AD}
\definecolor{defiblockbg}{HTML}{D9EDF7}
\newcommand{\defiblock}[2]{%
 \begin{varblock}{\textbf{Definition}}%
 {bg=defiblockbg,fg=defiblocktext}{bg=defiblocktext,fg=white}
  \begin{tabular}{ll}\textit{#1} & #2\end{tabular}
 \end{varblock} 
}

\definecolor{warnblocktext}{HTML}{C09853}
\definecolor{warnblockbg}{HTML}{D9EDF7}
\newcommand{\warnblock}[1]{%
 \begin{varblock}{\textbf{Warning!}}{bg=warnblockbg,fg=warnblocktext}{bg=warnblockbg,fg=warnblocktext}
  #1
 \end{varblock} 
}

\newcommand{\cout}[1]{%
 Output: \pbox[t]{\textwidth}{\ttfamily\fontsize{9}{10}\selectfont{}#1}
}

% CONSTANTS %%%%%%%%%%%%%%%%%%%%%%%%%%%


% Common title slide stuff
\title{Introduction to Programming with Scientific C++}
\author{Martin Uhrin}
\institute{UCL}
\date{November 7-9th 2012}

\subtitle{Session 2: Control structure}

\begin{document}

\frame{\titlepage}

\begin{frame}
\frametitle{Table of Contents}
\tableofcontents
\end{frame}

\section{Conditionals}

\begin{frame}[fragile]
	\frametitle{if, else}
	So far our programs have been linear: one start, one end, one path in between.  Let's branch out:
	\begin{lstlisting}
if(speed == 88)
  std::cout << "Great Scott!";
	\end{lstlisting}
	the statment after the \texttt{if} condition will only be executed if the condition is true.\pause  To execute more than one statement specify a block using \texttt{\{ \}}:
	\begin{lstlisting}
if(power >= 1.21)
{
  std::cout << "1.21 gigawatts!";
  std::cout << "1.21 gigawatts. Great Scott!\n";
}
	\end{lstlisting} 
	\pause
	We can also specify what to do if the condition is not fulfilled:
	\begin{lstlisting}
bool haveEnoughPower;
if(power >= 1.21)
  haveEnoughPower = true;
else
  haveEnoughPower = false;	
	\end{lstlisting}
\end{frame}



\begin{frame}[fragile]
  \frametitle{if, else}
  Lastly we can string together a series of conditionals, like so:
	\begin{lstlisting}
int width, height;
std::cin >> width >> height;
	
if(width > height)
  std::cout << "Fat rectangle\n";
else if(height > width)
  std::cout << "Tall rectangle\n";
else if(width == height)
  std::cout << "Square\n";
else
  std::cout << "IMPOSSIBLE!!";
	\end{lstlisting}
	These are evaluated one after another until one is found to be true, otherwise the final \kw{else} statement is executed.
	\pause
	\doblock{Keep conditionals simple, break them up if you have to.  A huge proportion of programming errors come from conditional statements.}
\end{frame}

\section{Loops}

\subsection{while loops}

\begin{frame}[fragile]
  \frametitle{Loops}
  The purpose of a loop is to execute a set of statements until a condition is fulfilled.
  \begin{block}{while loop}
    \begin{lstlisting}
while(expression) statement
    \end{lstlisting}
    Example:
  \begin{lstlisting}
  int t = 10;
  while(t != 0)
  {
    std::cout << t << ", ";
    --t;  
  }
  std::cout << "Blastoff!";
  \end{lstlisting}
  \end{block}
\end{frame}

\subsection{do-while loops}

\begin{frame}[fragile]
  \frametitle{Loops}
  \framesubtitle{Go on, go on, go on, go on, go on...}
  
  \begin{block}{do-while loop}
    \begin{lstlisting}
do statement while(expression);
    \end{lstlisting}
    \lstinputlisting[language=C++,title=\lstsrctitle]{../code/1_control_structure/lectures/tea_father.cpp}
  \end{block}
\end{frame}

\subsection{for loops}

\begin{frame}[fragile]
  \frametitle{Loops}
  
  \begin{block}{for loop}
    \begin{lstlisting}
  for(initialisation; condition; increment) statement;
    \end{lstlisting}
    This works as follows:
    \begin{enumerate}
      \item<2->{\texttt{initialisation} is executed.  Usually used to initialise a counter.  This happens once.}
      \item<3->{\texttt{condition} is checked, if true the loop continues, otherwise the loop ends and is skipped.}
      \item<4->{\texttt{statement} is executed.}
      \item<5->{\texttt{increment} is executed and we go back to step 2.}
    \end{enumerate}
    \pause[6]
    Here's our blastoff example with a for loop:
    \begin{lstlisting}
  for(int t = 0; t != 0; --t)
  {
    std::cout << t << ", ";
  }
  std::cout << "Blastoff!";
    \end{lstlisting}
    
  \end{block}
\end{frame}

\section{Functions}

\begin{frame}[fragile]
  \frametitle{Functions}
  \framesubtitle{How would we function without them?}
  Functions provide a way of structuring a program in a mode modular way, by grouping sets of statement so they can be easily reused.\newline
  \pause
  Consider: what if we want to calculate gravitational force between multiple bodies?
  \begin{lstlisting}
forceEarthSun = G * massEarth * massSun /
  (rEarthSun * rEarthSun);
forceEarthMoon = G * massEarth * massMoon /
  (rEarthMoon * rEarthMoon);
forceEarthMars = G * massEarth * massMars /
  (rEarthMars * rEarthMars);
  \end{lstlisting}
  \pause
  wouldn't it be better to be able to write:
  \begin{lstlisting}
forceEarthSun = force(massEarth,  massSun, rEarthSun);
forceEarthMoon = force(massEarth, massMoon, rEarthMoon);
forceEarthMars = force(massEarth, massMars, rEarthMars);
  \end{lstlisting}
  Through the magic of functions, we can!
\end{frame}


\begin{frame}[fragile]
  \frametitle{Functions}

  Format of a function:
  \begin{lstlisting}
type name(parameter1, parameter2, ...) { statements }  
  \end{lstlisting}
  where:
  \begin{itemize}
    \item<2->{\texttt{type} is the data type that is returned to you by the function.}
    \item<3->{\texttt{name} is the unique name of the function.}
    \item<4->{\texttt{parameter}, each of these is just like a variable definition (e.g. \texttt{\kw{double} mass}).}
    \item<5->{\texttt{statements} is the function's \textit{body}.  These statements will execute when the function is called.}
  \end{itemize}
  \pause[6]
  Let's try:
  \begin{lstlisting}
double force(const double mass1, const double mass2,
  const double r)
{
  return G * mass1 * mass2 / (r * r);
}
  \end{lstlisting}
\end{frame}


\begin{frame}[fragile]
  \frametitle{Functions}

  Let's try a full example:
  \lstinputlisting[language=C++,basicstyle=\ttfamily\fontsize{7}{8}\selectfont,title=\lstsrctitle]{../code/1_control_structure/lectures/grav_force.cpp}
\end{frame}

\begin{frame}
  \frametitle{Function nomenclature}
  
  \defiblock{caller}{A point in code that \textit{calls} a function.  A value is said to be \text{returned} to the caller when the function finishes.}
  
\end{frame}

\begin{frame}[fragile]
  \frametitle{Functions with no type}
  
  What if we want to have a function that doesn't return anything?\newline
  \pause
  Use \textit{void}:
  \begin{lstlisting}
void printForce(const std::string object1,
  const std::string object2,
  const double force)
{
  std::cout << object1 << " " << object2 << ":" << force << "\n";
}
  \end{lstlisting}
\end{frame}

\subsection{Declaring functions}

\begin{frame}[fragile]
  \frametitle{Declaring functions}
  \framesubtitle{I do declare}
  
  All identifier have to be need to be declared before they are used.  So how are we to code something like:
  \begin{lstlisting}
void setMicrowaveState(const bool newState) {
  if(newState == ON)
  {
    setMicrowaveDoor(CLOSED); // ERROR: Don't know about 
                              // setMicrowaveDoor
    state = ON;
  }
  else
  	state = OFF;
}
void setMicrowaveDoor(const bool newState) {
  if(newState == OPEN)
  {
    setMicrowaveState(OFF);
  }
  doorState = newState;
}
  \end{lstlisting}
\end{frame}


\begin{frame}[fragile]
  \frametitle{Declaring functions}
  \framesubtitle{I do declare}
  
  To get around this declare the \texttt{setMicrowaveDoor} function before \texttt{setMicrowaveState} like this:
  \begin{lstlisting}
void setMicrowaveDoor(const bool newState);

void setMicrowaveState(const bool newState) {
  if(newState == ON)
  {
    setMicrowaveDoor(CLOSED); // Happy: you've told me about
                              // setMicrowaveDoor
    state = ON;
  }
  ...
}
void setMicrowaveDoor(const bool newState) { /*as before*/ }
  \end{lstlisting}
  The first line tells the compiler: expect to see a \texttt{setMicrowaveDoor} function somewhere further down, and this is what it will look like.
\end{frame}

\begin{frame}[fragile]
  \frametitle{Declaring functions}
  TODO: Fix definition environment
  \begin{defiblocke}{function prototype}%
The header of a function without any of the body used in function declarations, e.g.:
    \begin{lstlisting}
void setMicrowaveDoor(const bool newState);
    \end{lstlisting}
  \end{defiblocke}
\end{frame}

\subsection{Default values}

\begin{frame}[fragile]
  \frametitle{Default values}
  \defiblock{default value}{A value that is used for a function parameter if the caller doesn't supply one.}
	An example:
\lstinputlisting[language=C++,title=\lstsrctitle,linerange={3-5,11,12-22},basicstyle=\ttfamily\fontsize{7}{8}\selectfont]{../code/1_control_structure/lectures/dispense_drink.cpp}  
\end{frame}

\begin{frame}[fragile]
  \frametitle{Default values}
  
  Default values have to be at the end of the parameter list, e.g.:
  \begin{lstlisting}
void dispenseDrink(int size,
  int drinkType = COFFEE,
  bool withMilk = false) { ... } // Good, can call:
  
dispenseDrink(1);      // or ...
dispenseDrink(3, TEA); // or ...
dispenseDrink(2, COFFEE, true)
  \end{lstlisting}
  \pause
  Bad:
  \begin{lstlisting}
void dispenseDrink(int drinkType = COFFEE,
  int size,
  bool withMilk = false) { ... }
  
dispenseDrink(/*what goes here?*/, 2); // Error!
  \end{lstlisting}
\end{frame}

\begin{frame}[fragile]
  \frametitle{Default values}
  
  \begin{doblocke}
    \begin{doitemize}
      \item{Use default values to automate commonly used parameter values or provide the user with optional parameters.}
      \pause
      \item{To give an indication of what a reasonable parameter value might be.}
    \end{doitemize}
  \end{doblocke}
\end{frame}

\subsection{Overloading functions}

\begin{frame}[fragile]
  \frametitle{Overloading functions}
  
  What if you want to create a dot product function that works for both integers and doubles?  Could write:
  \begin{lstlisting}
int dotInt(int x0, int y0, int x1, int y1)
{ return x0 * x1 + y0 * y1; }
double dotDouble(double x0, double y0, double x1, double y1)
{ return x0 * x1 + y0 * y1; }

dotInt(fromX, fromY, toX, toY);
  \end{lstlisting}
  Have to look up which dot function to call based on my number type. Ideally I'd like to call \texttt{dot} and let the compiler choose the right one.
  \newline\pause
  No problem, use:
  \begin{lstlisting}
int dot(int x0, int y0, int x1, int y1)
{ return x0 * x1 + y0 * y1; }
double dot(double x0, double y0, double x1, double y1)
{ return x0 * x1 + y0 * y1; }

dot(fromX, fromY, toX, toY); // Compiler will choose
                             // correct version based on
                             // from/to number types
  \end{lstlisting}
\end{frame}

\begin{frame}[fragile]
  \frametitle{Overloading functions}
  
  \defiblock{overloaded functions}{two or more functions with the same name but different parameter types and/or numbers of parameters.}
  Examples:
  \begin{lstlisting}
// Sum two or three integers
int sum(int n1, int n2);
int sum(int n1, int n2, int n3);

// Add together any integer/double
int add(int n1, int n2);
double add(int n1, double n2);
double add(double n1, int n2);
double add(double n1, double n2);
  \end{lstlisting}  

\end{frame}

\subtitle{Inbuild math functions}

\begin{frame}[fragile]
  \frametitle{Math functions}
  \framesubtitle{C numerics library}
  
  As scientists we're going to want to manipulate numbers.  Here are some commonly used functions that are available as part of the \texttt{cmath} header:
  
	  \begin{tabularx}{\linewidth}{XX}
		  \texttt{abs} & absolute value \\
		  \texttt{sin, cos, tan} & Warning: take angle in radians! \\
		  \texttt{exp, log, log10} & raise $e$ to power, natural log and base 10 log \\
		  \texttt{sqrt} & \\
		  \texttt{pow(\kw{double} base, \kw{double} exp)} & raise \texttt{based} to power \texttt{exp} 
	  \end{tabularx}
  \newline
  See \footnote{\href{http://www.cplusplus.com/reference/clibrary/cmath/}{\url{http://www.cplusplus.com/reference/clibrary/cmath/}}} for a full list.
  \newline
	\pause
  To use the math functions include the cmath header by writing:
  \begin{lstlisting}
#include <cmath>
  \end{lstlisting}
  at the start of you program.
\end{frame}

\subsection{Recursive functions}

\begin{frame}[fragile]
  \frametitle{Recursivity}
  \defiblock{recursive function}{A function that calls itself.}
  This is similar to a recurrence relation in mathematics e.g.:
  \begin{equation*}
  b_n = n b_{(n - 1)}, b_0 = 1
  \end{equation*}
  which gives the factorial of a number ($n!$).\pause{}  And in C++:
  \lstinputlisting[language=C++,title=\lstsrctitle,firstline=3,lastline=9]{../code/1_control_structure/lectures/factorial.cpp}  
  
\end{frame}

\end{document}
