\documentclass{beamer}
%\documentclass[handout]{beamer}

% Input all common stuff


% THEME SETTINGS %%%%%%%%%%%%
\usetheme[secheader]{Boadilla}
\usecolortheme{default}
\useinnertheme{circles}
\setbeamertemplate{blocks}[default]

% Get rid of bottom navigation bars
\setbeamertemplate{footline}[page number]{}

% Gets rid of navigation symbols
\setbeamertemplate{navigation symbols}{}

% PACKAGES %%%%%%%%%%%%%%%%%%%
\usepackage{listings}
\usepackage{xcolor}
\usepackage{multirow}
\usepackage{array}
\usepackage{verbatim}
\usepackage{pbox}

% LISTINGS STYLE %%%%%%%%%%%%%
\definecolor{bluekeywords}{rgb}{0.13,0.13,1}
\definecolor{greencomments}{rgb}{0,0.5,0}
\definecolor{redstrings}{rgb}{0.9,0,0}
\definecolor{listingsbg}{HTML}{EFEFFF}

\lstset{language=C++,
  showspaces=false,            % show spaces adding particular underscores
  showtabs=false,              % show tabs within strings adding particular underscores
  breaklines=true,             % sets automatic line breaking
  showstringspaces=false,      % underline spaces within strings
  breakatwhitespace=true,      % sets if automatic breaks should only happen at whitespace
  captionpos=b,                % sets the caption-position to bottom
  escapeinside={(*@}{@*)},
  commentstyle=\color{greencomments},
  keywordstyle=\color{bluekeywords}\bfseries,
  stringstyle=\color{redstrings},
  basicstyle=\ttfamily\fontsize{9}{10}\selectfont,
  backgroundcolor=\color{listingsbg},
}

% MACROS %%%%%%%%%%%%%%%%%%%%%
\newcommand{\lstsrctitle}[0]{%
 {\scriptsize\lstname}
}

\newcommand{\kw}[1]{\texttt{\textcolor{bluekeywords}{\ttfamily\fontsize{9}{10}\selectfont #1}}}

\newenvironment{varblock}[3]{%
  \setbeamercolor{block body}{#2}
  \setbeamercolor{block title}{#3}
  \begin{block}{#1}}{\end{block}}

% Colours taken from http://twitter.github.com/bootstrap/components.html#alerts
% Do block %%%%%%%%%%%%%%%%%%%%%%
\definecolor{doblocktext}{HTML}{468847}
\definecolor{doblockbg}{HTML}{DFF0D8}
\newenvironment{doblocke}[1]{% begin
  \begin{varblock}{\textbf{Do}}%
  {bg=doblockbg,fg=doblocktext}{bg=doblocktext,fg=white}%
  \lstset{backgroundcolor=}% The background should be the same as that of the block
  #1}%
  {\end{varblock}} %end
\newcommand{\doblock}[1]{%
 \begin{doblocke}#1\end{doblocke} 
}

% Don't block %%%%%%%%%%%%%%%%%%%
\definecolor{dontblocktext}{HTML}{B94A48}
\definecolor{dontblockbg}{HTML}{F2DEDE}  
\newenvironment{dontblocke}[1]{% begin
  \begin{varblock}{\textbf{Don't}}%
  {bg=dontblockbg,fg=dontblocktext}{bg=dontblocktext,fg=white}
  \lstset{backgroundcolor=}% The background should be the same as that of the block
  #1}%
  {\end{varblock}} %end

\newcommand{\dontblock}[1]{%
 \begin{varblock}{\textbf{Don't}}{bg=dontblockbg,fg=dontblocktext}{bg=dontblockbg,fg=dontblocktext}
 #1
 \end{varblock} 
}

\definecolor{defiblocktext}{HTML}{3A87AD}
\definecolor{defiblockbg}{HTML}{D9EDF7}
\newcommand{\defiblock}[2]{%
 \begin{varblock}{\textbf{Definition}}%
 {bg=defiblockbg,fg=defiblocktext}{bg=defiblocktext,fg=white}
  \begin{tabular}{ll}\textit{#1} & #2\end{tabular}
 \end{varblock} 
}

\definecolor{warnblocktext}{HTML}{C09853}
\definecolor{warnblockbg}{HTML}{D9EDF7}
\newcommand{\warnblock}[1]{%
 \begin{varblock}{\textbf{Warning!}}{bg=warnblockbg,fg=warnblocktext}{bg=warnblockbg,fg=warnblocktext}
  #1
 \end{varblock} 
}

\newcommand{\cout}[1]{%
 Output: \pbox[t]{\textwidth}{\ttfamily\fontsize{9}{10}\selectfont{}#1}
}

% CONSTANTS %%%%%%%%%%%%%%%%%%%%%%%%%%%


% Common title slide stuff
\title{Introduction to Programming with Scientific C++}
\author{Martin Uhrin}
\institute{UCL}
\date{November 7-9th 2012}

\subtitle{Session 5: Advanced object oriented programming}

\begin{document}

\frame{\titlepage}

\begin{frame}
\frametitle{Table of Contents}
\tableofcontents
\end{frame}

\section{Friendship}

\subsection{Friend functions}

\begin{frame}[fragile]
  \frametitle{Friend functions}
  \framesubtitle{I only give friends access to my privates}
  
  We've seen that only a class can access its \kw{private} members.  Sometimes we may want to make an exception for a particular external function.  We do this with the \kw{friend} keyword followed by the function prototype e.g.:
  \pause
  \lstinputlisting[language=C++,%
    basicstyle=\ttfamily\fontsize{8}{9}\selectfont,%
    title=\lstsrctitle,belowskip=0pt,
    linerange={3-12,25-28}]%
    {../code/5_advanced_oop/lectures/friend_function.cpp}

\end{frame}

\begin{frame}[fragile]
  \frametitle{Friend functions}
  \framesubtitle{Not for every day}
  
  \warnblock{The use of \kw{friend}s undermines encapsulation, after all you're giving away access to parts of your class that were designed to be kept internal.  Often this is a sign of bad design.
  \newline\pause
  Sometimes friendship can be used effectively to keep dependent code seaprated if there is a good design reason.  For example the code to draw a rectangle may be very complicated and may fit more naturally in the graphics handling portion of the codebase.
  }

\end{frame}

\subsection{Friend classes}

\begin{frame}[fragile]
  \frametitle{Friend classes}
  \framesubtitle{Not a taxonomy of friends}
  
  We can grant \kw{friend} status to an entire class:
  \begin{lstlisting}
  class SolarSystem {
  private:
    Vector2 planetPositions[NUM_PLANETS];
    Planet planets[NUM_PLANETS];
    friend class Planet;
  };
  
  class Planet {
  public:
    Vector2 getPosition()
    { return mySystem.planetPositions[myIndex]; }
  private:
    unsigned int myIndex;
    SolarSystem & mySystem;
  };
  \end{lstlisting}
  \pause
  Friend classes can be useful in highly coupled parts of code like this: a \texttt{Planet} cannot exist without a \texttt{SolarSystem};
  

\end{frame}

\section{Inheritance}

\begin{frame}
  \frametitle{Class inheritance}
  \framesubtitle{Keeping the poor in their place}

	Consider the following class:
  \lstinputlisting[language=C++,%
    basicstyle=\ttfamily\fontsize{8}{9}\selectfont,%
    linerange={4-19}]%
    {../code/5_advanced_oop/lectures/inheritance.cpp}
    \pause
  What if we want to create a dog class.  It can do everything a \texttt{Mammal} can.  It would be a shame to have to copy everything over.  With inheritance we don't have to!
\end{frame}

\begin{frame}
  \frametitle{Class inheritance}

  We can make a \texttt{Dog} inherit from \texttt{Mammal} like so:
  \lstinputlisting[language=C++,%
  	title=\lstsrctitle,belowskip=0pt,%
    linerange={54-65}]%
    {../code/5_advanced_oop/lectures/inheritance.cpp}
  \pause
  Now \texttt{Dog} has inherited members from \texttt{Mammal} so we don't have to rewrite them.  \texttt{Dog} is said to be \textit{derived} from \texttt{Mammal}. Makes sense, right?
\end{frame}

\begin{frame}[fragile]
  \frametitle{Class inheritance}
  \framesubtitle{Sit, boo-boo, sit.  Good dog.}
  
  This is what happens when we use the \texttt{Dog} class:
  \begin{columns}[t]
    \begin{column}[T]{0.5\linewidth}
  		\lstinputlisting[language=C++,%
  			title=\lstsrctitle,aboveskip=0pt,belowskip=0pt,%
    		linerange={86-102}]%
    		{../code/5_advanced_oop/lectures/inheritance.cpp}
    \end{column}
    \pause
  	\begin{column}[T]{0.48\linewidth}
  	  \cout{Fido is : 2 years old\newline
				and weighs 10 kg.\newline\newline
				Grrr, mammal noise!\newline
				\char`\\{}/\char`\\{}/\char`\\{}/ <- That's my tail wagging!\newline
				Fetching frisbee...Here you go.}
  	\end{column}
  \end{columns}

\end{frame}

\begin{frame}[fragile]
  \frametitle{Passing arguments to base constructors}
  
  \texttt{Mammal} has only one constructor:
  \lstinputlisting[language=C++,%
  	linerange={6-6}]%
  	{../code/5_advanced_oop/lectures/inheritance.cpp}
  So we need to give an age and a weight to build a \texttt{Mammal}.  Because \texttt{Dog} inherits from \texttt{Mammal} it too must provide these.\pause{} Here's how:
  \lstinputlisting[language=C++,%
  	linerange={67-73}]%
  	{../code/5_advanced_oop/lectures/inheritance.cpp}  
	The \texttt{Dog} class calls the \texttt{Mammal} constructor in the initialiser list as part of its constructor to initialise the \texttt{Mammal} part of itself.

\end{frame}

\begin{frame}[fragile]
  \frametitle{Initialiser lists}
  
	\defiblock{initialiser list}{a list used to initialise base classes, class constants, references and (optionally) member variables as part of a class's constructor.}
	
	The format is:
	\begin{lstlisting}
  constructor_name(paramters...):
    initialise_item1,
    initialise_item2,
	  ...
	  { /* constructor body */ }
	\end{lstlisting}
	So we could also initialise a \texttt{Dog}'s breed in this list:
	\begin{lstlisting}
	Dog(const unsigned int age,
	  const unsigned int weight,
	  const std::string & breed)
	  myBreed(breed) {}
	\end{lstlisting}
\end{frame}

\begin{frame}[fragile]
  \frametitle{Overriding functions}
  
  You may have noticed that when \texttt{fido} spoke he said ``\texttt{Grrr, mammal noise!}'', not really what a Dog says.  That's because I didn't provide a custom \texttt{speak} method.  Let's try again:\pause
  \begin{columns}[t]
    \begin{column}[T]{0.55\linewidth}
  		\lstinputlisting[language=C++,%
  			title=\lstsrctitle,aboveskip=0pt,
  			linerange={54-55,62-63,67-67,86-94}]%
  			{../code/5_advanced_oop/lectures/inheritance2.cpp}  
  	\end{column}
  	\begin{column}[T]{0.4\linewidth}
  	  \cout{Woof!}
  	\end{column}
  \end{columns}
  \pause
  The new \texttt{speak} method is said to \textit{override} the one in \texttt{Mammal}.

\end{frame}

\begin{frame}[fragile]
  \frametitle{Derived class member access}
  
  Finally we get to see what the \kw{protected} access specifier is all about.  Access types can be summarised as follows:
  \begin{table}[h]
    \centering
	  \begin{tabularx}{0.75\linewidth}{l|c|c|c}
	    Access & \kw{public} & \kw{protected} & \kw{private} \\
	    \hline
	    members of same class & yes & yes & yes \\
	    members of derived class & yes & yes & no \\
	    non members & yes & no &  no
	  \end{tabularx}
  \end{table}
  
  \doblock{Use the most restrictive access possible.  Member variables should rarely be anything but \kw{private}, even derived classes should use getters/setters to access these.  It's find to make methods protected though.}
 
\end{frame}

\begin{frame}[fragile]
  \frametitle{What \textit{exactly} is inherited?}
  
  Pretty much everything is inherited by the derived class, except:
  \begin{itemize}
    \item{Constructors and destructors}
    \item{\texttt{operator =()} members}
    \item{friends}
  \end{itemize}

\end{frame}

\end{document}